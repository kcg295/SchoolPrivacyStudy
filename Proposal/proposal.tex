%%%%%%%%%%%%%%%%%%%%%%%%%%%%%%%%%%%%%%%%%
% Journal Article
% LaTeX Template
% Version 1.4 (15/5/16)
%
% This template has been downloaded from:
% http://www.LaTeXTemplates.com
%
% Original author:
% Frits Wenneker (http://www.howtotex.com) with extensive modifications by
% Vel (vel@LaTeXTemplates.com)
%
% License:
% CC BY-NC-SA 3.0 (http://creativecommons.org/licenses/by-nc-sa/3.0/)
%
%%%%%%%%%%%%%%%%%%%%%%%%%%%%%%%%%%%%%%%%%

%----------------------------------------------------------------------------------------
%	PACKAGES AND OTHER DOCUMENT CONFIGURATIONS
%----------------------------------------------------------------------------------------

\documentclass[twoside,twocolumn]{article}

\usepackage{blindtext} % Package to generate dummy text throughout this template 

\usepackage[sc]{mathpazo} % Use the Palatino font
\usepackage[T1]{fontenc} % Use 8-bit encoding that has 256 glyphs
\linespread{1.05} % Line spacing - Palatino needs more space between lines
\usepackage{microtype} % Slightly tweak font spacing for aesthetics

\usepackage[english]{babel} % Language hyphenation and typographical rules

\usepackage[hmarginratio=1:1,top=32mm,columnsep=20pt]{geometry} % Document margins
\usepackage[hang, small,labelfont=bf,up,textfont=it,up]{caption} % Custom captions under/above floats in tables or figures
\usepackage{booktabs} % Horizontal rules in tables

\usepackage{lettrine} % The lettrine is the first enlarged letter at the beginning of the text

\usepackage{enumitem} % Customized lists
\setlist[itemize]{noitemsep} % Make itemize lists more compact

\usepackage{abstract} % Allows abstract customization
\renewcommand{\abstractnamefont}{\normalfont\bfseries} % Set the "Abstract" text to bold
\renewcommand{\abstracttextfont}{\normalfont\small\itshape} % Set the abstract itself to small italic text

\usepackage{titlesec} % Allows customization of titles
\renewcommand\thesection{\Roman{section}} % Roman numerals for the sections
\renewcommand\thesubsection{\roman{subsection}} % roman numerals for subsections
\titleformat{\section}[block]{\large\scshape\centering}{\thesection.}{1em}{} % Change the look of the section titles
\titleformat{\subsection}[block]{\large}{\thesubsection.}{1em}{} % Change the look of the section titles

\usepackage{fancyhdr} % Headers and footers
\pagestyle{fancy} % All pages have headers and footers
\fancyhead{} % Blank out the default header
\fancyfoot{} % Blank out the default footer
\fancyhead[C]{PrivacyLx $\bullet$ Junho 2020 $\bullet$ Proposta de Projeto} % Custom header text
\fancyfoot[RO,LE]{\thepage} % Custom footer text

\usepackage{titling} % Customizing the title section

\usepackage{hyperref} % For hyperlinks in the PDF

%----------------------------------------------------------------------------------------
%	TITLE SECTION
%----------------------------------------------------------------------------------------

\setlength{\droptitle}{-4\baselineskip} % Move the title up

\pretitle{\begin{center}\Huge\bfseries} % Article title formatting
\posttitle{\end{center}} % Article title closing formatting
\title{Measuring Student Privacy} % Article title
\author{%
\textsc{Kevin Gallagher}\\%\thanks{A thank you or further information} \\[1ex] % Your name
\normalsize PrivacyLx \\ % Your institution
\normalsize \href{mailto:contact@privacylx.org}{contact@privacylx.org} % Your email address
%\and % Uncomment if 2 authors are required, duplicate these 4 lines if more
%\textsc{Jane Smith}\thanks{Corresponding author} \\[1ex] % Second author's name
%\normalsize University of Utah \\ % Second author's institution
%\normalsize \href{mailto:jane@smith.com}{jane@smith.com} % Second author's email address
}
\date{\today} % Leave empty to omit a date
\renewcommand{\maketitlehookd}{%
\begin{abstract}
\noindent A educação é um direito humano fundamental e é necessária para que um indivíduo possa interagir com a sociedade de forma significativa e produtiva. A educação, porém, não deve ser feita à custa de outros direitos humanos, como a privacidade. Neste projecto propomos determinar até que ponto o sistema educativo português obriga os alunos a renunciar ao seu direito à privacidade, enviando dados dos alunos a plataformas de terceiros, ou forçando os alunos a com elas interagir. Propomo-nos a fazê-lo através do levantamento de múltiplas fontes: direções escolares, responsáveis de proteção de dados das escolas, técnicos informáticos escolares e pais de alunos do sistema escolar português. Estes dados serão então analisados para determinar quantas plataformas de terceiros estão a receber esta informação, e a quantidade de informação que estão a receber dos ou sobre os alunos.
\end{abstract}
}

%----------------------------------------------------------------------------------------

\begin{document}

% Print the title
\maketitle

%----------------------------------------------------------------------------------------
%	ARTICLE CONTENTS
%----------------------------------------------------------------------------------------

\section{Introdução}


A educação é um direito humano fundamental e é necessária para que um indivíduo possa interagir com a sociedade de forma significativa e produtiva. Através da educação, os estudantes podem aprender competências fundamentais como a matemática, a ciência e o pensamento crítico. Podem aprender factos necessários para navegar em burocracia e sistemas complexos. Podem aprender as competências necessárias para o mundo de trabalho, ou para criarem a sua própria empresa. Bem como aprender a importância da sua participação no âmbito dos sistemas políticos e económicos. A educação é claramente importante para o desenvolvimento das crianças à medida que estas aprendem a tornar-se membros da sociedade.


No entanto, a educação não deve ser feita à custa de outros direitos humanos, como a privacidade. As técnicas educativas modernas utilizam a tecnologia moderna com grande eficácia, como por exemplo permitindo uma comunicação rápida entre professor e aluno fora da sala de aula, facilitando uma partilha rápida de ficheiros entre alunos, ou permitindo formas inovadoras e interessantes de interacção dos alunos com o material de aprendizagem. Infelizmente, muitas das empresas que oferecem estes métodos de aprendizagem operam como empresas com modelos de negócio que dependem da retenção e venda de dados dos utilizadores. Quando as escolas adoptam estas tecnologias pelas razões mais benignas, partilham os dados dos seus alunos com estas empresas, que podem então utilizá-los para fins publicitários. Mesmo que não enviem os dados dos seus alunos a estas empresas, se obrigarem os seus alunos a interagir com ferramentas ou serviços que estas empresas oferecem, estas empresas podem utilizar JavaScript, plugins ou outras tecnologias para recolher informação sobre as identidades, comportamentos e interesses dos alunos. 


Preservar a privacidade dos alunos não implica, contudo, que a aprendizagem e o ensino devam ser feitos com técnicas antiquadas. Há uma multiplicidade de maneiras como as escolas e o sistema educativo no seu conjunto pode proteger a privacidade dos alunos, destacando-se particulamente a hospedagem pela prórpia escola ou agrupamento de software de teleconferência, os sistemas de gestão da aprendizagem e a utilização da sua própria infra-estrutura privada para armazenamento de ficheiros. Embora tudo isto seja possível, muitas escolas e sistemas educativos optam por não o fazer devido a limitações financeiras, em vez de considerarem os direitos dos seus alunos.


\section{Proposta}
\label{sec:proposal}

Propomo-nos analisar até que ponto as escolas portuguesas violam a privacidade dos seus alunos ao exporem os seus dados a empresas terceiras, ou ao exigirem que os seus alunos interajam com serviços de terceiros. \footnote{Embora muitas escolas possam ter acordos de restrição de dados com estes serviços de terceiros, não consideramos que este seja um argumento válido para afirmar que a privacidade é preservada. Alguns alunos podem não querer que os dados sejam transmitidos a terceiros, tornando a transmissão uma violação da privacidade. Adicionalmente, consideramos que tais acordos para evitar a venda e partilha são impossíveis de fornecer garantias e, portanto, não merecem ser considerados.} Especificamente, para cada escola em Portugal, pretendemos responder às seguintes questões:


\begin{enumerate}
\itemsep0em
\item A escola transmite informações sobre os alunos a terceiros?
\item Quais os terceiros que estão a receber dados dos alunos?
\item Quais são os métodos utilizados para proteger os dados transmitidos?
\item Porque é que os dados são transmitidos a terceiros?
\item A escola exige que os alunos utilizem serviços de terceiros?
\item Quais são os terceiros com quem os alunos são obrigados a interagir?
\item Estes serviços contêm trackers que enviam dados a ainda mais terceiros?
\item Porque é que a escola exige que os alunos utilizem estes serviços de terceiros?

\end{enumerate}

A recolha desta informação permitir-nos-á compreender melhor a extensão da violação da privacidade dos alunos em Portugal, o que nos pode ajudar a melhorar o estado actual das coisas (ver secção \ref{sec:deliverables}).

%------------------------------------------------

\section{Métodos}

Propomos responder às questões mencionadas na secção \ref{sec:proposal} através da quatro inquéritos diferentes para cada escola, sempre que possível. Cada inquérito será enviado a pessoas diferentes papeis no funcionamento da escola. Por exemplo, uma pergunta sobre se há trackers em serviços de terceiros é melhor colocada ao profissional de TI de uma escola, e não ao presidente. Iremos criar inquéritos para os seguintes intervenientes de uma escola:

\begin{itemize}
\itemsep0em
\item Direção
\item Profissionais de TI (quando possível)
\item Responsável pela privacidade (quando possível)
\item Pais de alunos, ou presidente de associação de pais (quando possível)
\end{itemize}


Para facilitar a análise, estes inquéritos incluirão principalmente perguntas de escolha múltipla e algumas perguntas abertas. As perguntas abertas serão analisadas pelo pessoal disponível para descobrir que perigos adicionais à privacidade podem existir para os alunos e para descobrir as motivações por detrás das decisões que põem em risco a privacidade dos alunos.

%------------------------------------------------

\section{Documentos a produzir}
\label{sec:deliverables}

Após a recolha e análise destes dados, pretendemos publicar as estatísticas agregadas das nossas conclusões, bem como um relatório de sumário dos resultados obtidos. Pretendemos também incluir no nosso relatório potenciais soluções sobre como as escolas ou o Ministério da Educação podem avançar no sentido de proteger os direitos de privacidade dos alunos em Portugal.


\section{Conclusões}

A educação em Portugal não deve ser feita à custa do direito à privacidade. Nesta proposta apresentamos uma série de perguntas que pretendemos responder e discutimos a recolha desta informação através de quatro inquéritos separados. A partir destes inquéritos iremos produzir estatísticas agregadas, um relatório de síntese das nossas conclusões e uma lista de recomendações às escolas e ao Ministério da Educação para ajudar a proteger melhor os direitos dos alunos à privacidade.

%------------------------------------------------
%\section{Results}
%------------------------------------------------

%\section{Discussion}




%----------------------------------------------------------------------------------------
%	REFERENCE LIST
%----------------------------------------------------------------------------------------


%----------------------------------------------------------------------------------------

\end{document}
